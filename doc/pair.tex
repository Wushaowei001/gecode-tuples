\begin{figure}[H]
  \centering
  \pgfplotsset{
    scale only axis,
%    xmin=1,
  }
  \begin{tikzpicture} [scale=0.7]
    \begin{axis}[
        title=IntPairExact,
        xtick=data,
        ymin=0,
      ]
      \addplot[black, mark=*, only marks]
       plot coordinates
       {  
       (1,1)       (3,1) (4,1) (5,1) (6,1) (7,1)
       (1,2) (2,2) (3,2) 
             (2,3) (3,3) (4,3) (5,3)
       (1,4)       (3,4)       (5,4)
             (2,5)        (4,5) 
       (1,7)                               (7,7)
       }; 
    \end{axis}
  \end{tikzpicture}
  \begin{tikzpicture} [scale=0.7]
    \begin{axis}[
        yticklabel pos=right,
        xtick=data,
        title=IntPairApprox,
        ytick=data
        ]
      \addplot[black, mark=*, only marks]
       plot coordinates
       {  
       (1,1)       (3,1) (4,1) (5,1) (6,1) (7,1)
       (1,2) (2,2) (3,2) 
             (2,3) (3,3) (4,3) (5,3)
       (1,4)       (3,4)       (5,4)
             (2,5)        (4,5) 
       (1,7)                               (7,7)
       }; 
      \addplot[red, mark=*, only marks]
       plot coordinates
       {  
             (2,1) 
       


                  (2,4)    (4,4)
                       (3,5)
            (2,7) (3,7) (4,7) (5,7) (6,7)
       }; 

    \end{axis}
  \end{tikzpicture}

 \caption{Comparasion between IntPairExact and IntPairApprox representing the same domain. The red dots are values that should not be there, but IntPairApprox only stores the boundaries for the second dimension.}\label{fig:pair}
\end{figure}
